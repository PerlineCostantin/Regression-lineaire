\documentclass[12pt]{article}

%%    LES PACKAGES
\usepackage{import}
\usepackage{enumitem}%Pour faire des listes à puces stylés.
\usepackage{pifont} %Pour changer les symboles dans les listes à puces.
\usepackage{stmaryrd}%Pour les intervalles discrets.
\usepackage{amsmath}
\usepackage{amsfonts}
\usepackage{amssymb}
\usepackage{amsthm}
\usepackage{esint}
\usepackage[most,many,breakable]{tcolorbox}
\usepackage{graphicx}
\usepackage{color}
\usepackage{nicefrac}
\usepackage{mathtools}
\usepackage{xcolor}
\usepackage{awesomebox}
\usepackage{cancel}
\usepackage{setspace}
\usepackage{varwidth}
\usepackage{mathrsfs}
\usepackage[frenchb]{babel}
\usepackage[left=1cm, right=1cm, top=2cm, bottom=2cm]{geometry}

%%    MISE EN PAGE





\definecolor{Aquamarine}{HTML}{00B5BE}
\DeclareMathSizes{10}{9}{7}{5}
\baselineskip=12pt


%%%%%%%%%%%%%%%%%%%%%%%%%%%%%%
%%    LES ENVIRONNEMENTS
%%%%%%%%%%%%%%%%%%%%%%%%%%%%%%


\newtheorem{definition}{Définition}[section]
\newtheorem{propriete}{Propriétés}
\newtheorem{theoreme}{Théorème}
\newtheorem*{proposition}{Proposition}
\newtheorem*{preuve}{Preuve}
\newtheorem*{remark}{Remarque}
\newtheorem{exercice}{Exercice}
\newcommand{\defi}{\begin{tcolorbox}[title=Définition, colframe=red!75!black, colback=red!10!white]}
\newcommand{\prop}{\begin{tcolorbox}[title=Propriété, colframe=blue!85!black, colback=blue!10!white]}
\newcommand{\impo}{\begin{tcolorbox}[title=Points importants, colframe=black!0!black, colback=black!0!white]}
\newcommand{\close}{\end{tcolorbox}}
\newtcolorbox{mybox}{colback=black!5!white,colframe=black!75!black}


%% LES MACROS
\newcommand{\summ}[2]{\sum_{#1}^{#2}}
\newcommand{\dps}{\displaystyle}
\newcommand{\sk}{\smallskip}
\newcommand{\bk}{\bigskip}
\newcommand{\jump}{~~}
\newcommand{\acc}[1]{\hspace{-0.05cm}\left\{#1 \right\}}
\newcommand{\parr}[1]{\hspace{-0.05cm}\left(#1 \right)}
\newcommand{\rect}[1]{\left[ #1 \right]}
\newcommand{\braks}[1]{\hspace{-0.05cm}\llbracket#1 \rrbracket}
\newcommand{\abs}[1]{\hspace{-0.05cm}\left\lvert#1 \right\rvert}
\newcommand{\eq}{\Leftrightarrow}
\newcommand{\mc}[1]{\mathcal{#1}}
\newcommand{\mb}[1]{\mathbb{#1}}
\newcommand{\mr}[1]{\mathscr{#1}}
\newcommand{\Ra}{\Rightarrow}
\newcommand{\projet}[3]{p_{#1\Vert #2}(#3)}
\newcommand{\projeet}[2]{p_{#1\Vert #2}}
\newcommand{\ra}{\rightarrow}
\newcommand{\lra}{\longrightarrow}
\newcommand{\la}{\leftarrow}
\newcommand{\exs}{\exists}
\newcommand{\fonction}[3]{#1:#2\longrightarrow#3}
\newcommand{\famille}[3]{#1_#2,\dots,#1_#3}
\newcommand{\transpose}[1]{{}^{t\!}#1}
\newcommand{\longvec}[1]{\overrightarrow{#1}}

%% MACRO POUR LES MBOX
\newcommand{\tw}{\textwidth}
\newcommand{\blue}{\color{blue}}
\newcommand{\red}{\color{red}}
\newcommand{\green}{\color{green}}






\title{Projet MAM3 : Régression linéaire}
\author{Ben Khalifa Emna, Costantin Perline, Honakoko Giovanni}
\date{26/05/2025}

	\setlength{\intextsep}{12pt plus 2pt minus 2pt} % Espace autour des figures
\setlength{\textfloatsep}{12pt plus 2pt minus 2pt} % Espace entre figures
\setlength{\parindent}{0pt} % Supprime l'indentation de tous les paragraphes


\begin{document}
	\maketitle

	\newpage
	\tableofcontents
	
	\section{Théorie}
	\subsection{Cadre}
	On se place dans le cadre de la régression linéaire simple où on a une variable réponse et une variable explicative qui sont quantitatives.
	On dispose de $\mc{L} := \acc{(x_{i},y_{i})_{i \in \braks{1,n}}}$ où :
	\begin{itemize}[label*=\textbullet]
		\item $i$ représente l'individu considéré 
		\item $x_{i}$ représente les observations de la variable explicative
		\item $y_{i}$ représente les observations de la variable réponse
	\end{itemize}
	On cherche $f$ la fonction telle que : $\forall i \in \braks{1,n}, y_i \approx f(x_i)$. Pour estimer la fonction $f$ on veut minimiser le risque quadratique : 
	\begin{equation*}
		R(g) := \mb{E}[\parr{Y- g(X)}^{2}]
	\end{equation*}
	où $Y$ est la variable réponse et $X$ est la variable explicative. 
	Une estimation de $f$ est :
	\begin{equation*}
		f^{*} := \underset{g}{\mathrm{argmin}}(g)
	\end{equation*}
	Cette quantité étant purement théorique on l'a substitue à sa quantité empirique le risque empirique $R_{n}(g)$ qui s'exprime comme: 
	\begin{equation*}
		R_{n}(g) := \dfrac{1}{n}\sum_{i=0}^{n}\parr{Y_{i} -g(X_i)}^{2}
	\end{equation*}
	
	On supposera que $g$ appartient à l'ensemble $\mc{F} = \acc{g : \mb{R}\lra \mb{R} , \hspace{2mm} g(x)= ax + b, \forall a,b \in \mathbb{R}}$.
	
	Dans notre cadre $Y_{i} = ax_{i}+ b + \varepsilon_{i}$ où $\varepsilon_i$ représente le bruit pour l'individu $i$. Les $Y_{i}$ et les $\varepsilon_{i}$ sont des quantités aléatoires contrairement aux $x_{i}$ qui sont fixes.
	
	
	\subsection{Estimateurs paramétriques $\hat{a}_n$ et $\hat{b}_n$}
	On pose : 
	\begin{equation*}
		\overline{x}_{n} = \frac{1}{n}\sum_{i=1}^{n}x_i\quad , \quad  \overline{Y}_{n} = \frac{1}{n}\sum_{i=1}^{n}Y_i
	\end{equation*}
	Pour cela on cherche les points critiques : 
	\begin{align*}
		&\begin{cases}
			\dps 
			\frac{\partial R_{n}(g)}{\partial a} = 0
			\\
			\dps \frac{\partial R_{n}(g)}{\partial b} = 0
		\end{cases}
		\\
		&\begin{cases}
			\dps 
			-\frac{2}{n}\sum_{i=1}^{n}x_i \parr{Y_i - ax_i - b } = 0
			\\
			\dps -\frac{2}{n}\sum_{i=1}^{n}(Y_i - ax_i - b) = 0
		\end{cases}
		\\
		&\begin{cases}
			\dps 
			\sum_{i=1}^{n} x_i Y_i - a\sum_{i=1}^{n}x_i^{2} - b\sum_{i=1}^{n}x_i = 0
			\\
			\dps b = \overline{Y}_{n} - a\overline{b}_n
		\end{cases}
	\end{align*}
	
	En réinjectant l'expression de $b$ dans la première ligne on obtient :
	\begin{align*}
		\sum_{}^{}x_iY_i - \overline{Y}_n\sum_{i=1}^{n}x_i - a\overline{x}_n\sum_{i=1}^{n}x_i &= 0
		\\
		\sum_{i=1}^{n}x_i Y_i - \overline{Y}_{n} \sum_{i=1}^{n}x_i &= a \rect{ \sum_{i=1}^{n}x_i^{2} - \overline{x}_n \sum_{i=1}^{n}x_i }
		\\
		\sum_{i=1}^{n}x_i Y_i -n\overline{y}_n \overline{x}_n&= a \rect{\sum_{i=1}^{n}x_i^2 -n\parr{\overline{x}_n}^2} 
		\\
		a&=\dfrac{\sum_{i=1}^{n}x_i Y_i -n\overline{Y}_n \overline{x}_n}{\sum_{i=1}^{n}x_i^2 -n\parr{\overline{x}_n}^2}
	\end{align*}
On a bien :
\begin{equation*}
	\begin{cases}
		a=\dfrac{\sum_{i=1}^{n}x_iY_i -n\overline{Y}_n \overline{x}_n}{\sum_{i=1}^{n}x_i^2 -n\parr{\overline{x}_n}^2}
		\\
		b=\overline{Y}_n - a\overline{x}_{n}
	\end{cases}
\end{equation*}
\ding{110}

\subsection*{Proposition}
Les estimateurs $\hat{a}_{n}$ et $\hat{b}_{n}$ sont sans biais.

\textsc{Preuve}

\end{document}